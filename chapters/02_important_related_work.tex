% !TeX root = ../main.tex
% Add the above to each chapter to make compiling the PDF easier in some editors.

\chapter{Related Work}\label{chapter:related_work}

% \section{Section}
% Review studies that have applied topic modeling
The recent advances in \ac{NLP} and easy access to open-source models allow researchers to 
study text data by performing sentiment and emotional analysis, topic modeling, 
semantic search, and many more. Large language models like ChatGPT by OpenAI considerably 
explain how fast the NLP field develops.

In this thesis, topic modeling is performed on massive text data. Topic modeling is an 
unsupervised tool that helps extract the underlying themes from the given text data. 
There are several topic modeling approaches, and this thesis focuses on neural topic modeling. 
Unlike conventional models like Latent Dirichlet Allocation (LDA), a generative probabilistic 
model \parencite{blei_lda_2003}, neural topic models have been used in important NLP 
tasks, including text generation, document summarisation, and translation, fields to 
which conventional topic models are complex to apply \parencite{zhao_neural_topic_models_2021}.
This thesis uses the neural topic model BERTopic, introduced by \textcite{bertopic}, 
which is explained in detail in the following chapters.

A tremendous number of studies have applied topic modeling in their research. 
In the political science field, \textcite{ilyas_brexit_topic_modeling_2020} performed 
topic modeling using LDA to evaluate Brexit’s impact in the UK.
\textcite{kaiser_right_media_USA_stm_topic_modeling_2020} used a structural topic model (STM), 
similar to LDA, to analyze the right media coverage during the 2016 US elections, while 
\textcite{anwar_analyzing_twitter_BERT_QAnon_2021} performed topic modeling using BERT on 
Twitter data for 2020 US elections. \textcite{bertopic_twitter_german_politics_2022} 
applied BERTopic on Twitter data from German politicians and analyzed its results. 
\textcite{contreras_panama_lda_bertopic_2022} used both LDA and BERTopic on a 
Panamanian parliamentary proceedings, which are Spanish. 
% 2. Political Discourse on Twitter
% role of Twitter in political campaigns, elections, and public discourse worldwide
% Turkey's case

% 3. Comparative Analyses
%  include studies that offer a comparative analysis 

% 4.Identify Gaps and Opportunities
% gaps in the existing literature, e.g. for Turkey
% Justify the research, by explaining the gap

% 5. Methodological Considerations
% strengths and limitations, why BERTopic?

% 6. End: contribution of this thesis to existing literature
