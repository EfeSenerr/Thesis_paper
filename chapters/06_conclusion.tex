% !TeX root = ../main.tex
% Add the above to each chapter to make compiling the PDF easier in some editors.

\chapter{Conclusion}\label{chapter:conclusion}

In conclusion, this thesis analyzes 150 million tweets from July 2022 to June 2023 to discover 
frequently discussed topics during the months up to the election. Since the academic API no longer 
functions, the analyzed tweets have been collected from the \#Secim2023 dataset published by 
\textcite{secim2023}. BERTopic, a neural topic model, has been used for topic modeling. A base 
model has been trained using a random sample of 1\%, and the rest of the data has been transformed 
to the corresponding topic results of the base model.

Two different analysis strategies have been used. The results contain more than 1000 topic clusters for 
the first and more than 7000 for the second approach. The top 15 trending topics of the results and 
their corresponding representative words can be found in 
\autoref{tab:topic_modeling_results_1} and \autoref{tab:topic_modeling_results_2}, and the remaining 
detailed information is on the affiliated GitHub page of the 
repository\footnote{\url{https://github.com/EfeSenerr/Thesis}}. The resulting top 15 topics include general 
political discussions, praise and criticism of both ruling and opposition parties, wishes and demands from 
the government, and the earthquake disaster. In \autoref{chapter:results}, the results have been 
analyzed in detail and divided into three main categories for better interpretation. These categories 
are the ruling party, opposition, and other related topics. For each category, the topics are 
visualized to interpret the topic trends on a normalized monthly basis. 

In the analysis, the most significant topic cluster emerged, identified as 
\textit{Political Discussions}, has been used as the baseline topic and includes tweets with positive 
and negative sentiments towards both the ruling and the opposition parties. Over one-tenth of the 
tweets are in this topic cluster in the election month. 

The topics related to the government lean toward nationalism, criticism and praise of political 
figures, and Erdoğan's chance of winning the elections.
On the other hand, the topics related to the opposition lean toward the discussions around opposition 
figures and 
presidential candidates, the Nation Alliance coalition and criticism against them, treason accusations,
and provocations. For instance, looking at the increasing trend of the opposition presidential 
candidacy in the \autoref{fig:topics_graph_opposition}, one can notice how the topic gains importance 
throughout the months. 

The thesis highlights the polarized political discourse and discovers essential topics for Twitter 
users by analyzing the trends of the topic clusters. The topics that are not related to the previous 
two categories include topics like the earthquake disaster, religious wishes, demands from the 
government related to teachers, and the retirement system.

Moving forward, two additional visualizations were presented to analyze the remaining topic clusters 
in detail, which were not discussed earlier. The first shows the topics in a two-dimensional space, 
and the second displays the hierarchical clustering of the top 50 topics. Various topic clusters were 
noticed and discussed. For instance, the war between Russia and Ukraine, demands from the government 
in regards to military sergeants and agriculture engineers, refugee policy, problems caused by 
free-roaming dogs, political figures like Meral Akşener, nationalism, Atatürk, and accusation of 
parties to having a relationship with terrorist organizations. With this, the thesis answers the first 
research question, which aims to discover the most prevalent Twitter topics during the May election.

After that, the topic model results have been analyzed in a broader concept, highlighting the 
correlation between real-life events and the shifts in discussions on Twitter. The top 1000 topics 
have been intensively interpreted and picked for analysis, where the thesis emphasizes the mirror 
between the trending topics on social media and the parties' agendas and public statements on both 
ruling and opposition ranks. The detailed focus starts with the opposition ranks and continues with 
the ruling block. The election agendas, strategies, and mottos of parties and political figures have 
been highlighted for each focus. These are then compared with their corresponding topic clusters, 
underlining their similarities and differences. 

Furthermore, besides the agendas and strategies, the focus on opposition rank covers miscalibration 
amongst the Nation Alliance and the incapability to select an opposition candidate, unlawful and 
biased use of judiciary against the opposition block, the hyperinflation and the economic situation, 
and the disinformation and fake news campaign against the opposition block. On the other hand, the 
focus on the ruling rank covers the success rate of the ruling party in the polls and then the 
elections by highlighting the foreign policy, the President Erdoğan, the second powerful party in 
the People Alliance \ac{MHP} and nationalism,  the problem of the merge of the state and the ruling 
party, emphasis on religion, and the theme of continuity and stability.

The thesis continues by comparing the results with other research that has research questions similar 
to this thesis. The key themes, content, and engagement levels of the resulting topics are considered 
in the comparison, which answers the last research question of this thesis.
First, the focus lies on research relevant to Turkish elections. The topic model results are compared 
with those of other research that also analyzed Twitter data but focused on other methods, such as 
sentiment and emotional analysis. 
Next, the comparison continues with research that leveraged the BERTopic and other topic models on 
different political domains. The thesis highlights the similarities in relevant research findings, 
such as the correlation between real-life events and trending topics on Twitter.

The discussion continues with the limitations of this thesis. The analyzed dataset is partially 
collected in this thesis due to the lack of academic API and the deletion of tweets. The limitations 
of the Twitter data are also highlighted, mentioning the creation of trending topics by automated 
attacks, the high percentage of rebroadcasting information, and the low number of users that post 
almost all the tweets.

The second mentioned limitation covers the leveraged topic model, BERTopic. The performance of 
BERTopic varies slightly due to the dataset's multilingualism and varies significantly by adjusting 
the parameters. The embedding approach of BERTopic results in a massive number of resulting topics, 
which requires effort to examine each topic. In addition, labor-intensive analysis of each topic 
prevents objective evaluation of the topics.

Finally, various future works of the thesis are highlighted. In the short period, one could change 
the parameters of BERTopic, try a different topic modeling strategy or embedding model, or reduce 
the number of topics to obtain better topic clusters. Besides topic modeling, sentiment and emotional 
analysis could be conducted, leveraging pre-trained language models. That could help determine 
whether the sentiment toward specific topics correlates with the shifts in election voting patterns.

In the long term, other methods exist to interpret and analyze the May 2023 election. One could 
scrape and analyze traditional media platforms or other platforms like Ekşi Sözlük to gain a different 
perspective and see how their trending topics and sentiments vary between platforms. 
The analysis focus could also be directly on politicians' speeches, parties' public announcements, 
and parliament discussions, which could give insights into better interpreting the political figures 
and parties.

In short, this thesis contributes to the academic environment by pioneering one of the first 
topic modeling analyses on Turkish political discourse utilizing big data, which sheds light on 
the May 2023 general elections. This thesis aims to better understand the Turkish political discourse 
and motivate future research by presenting the tested methodologies and findings.