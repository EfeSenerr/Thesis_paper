% !TeX root = ../main.tex
% Add the above to each chapter to make compiling the PDF easier for some editors.

\chapter{Introduction}\label{chapter:introduction}

In recent years, governments and the public have realized the importance of social media, 
especially Twitter/X, which has a decisive role in mobilizing social and political activism 
\parencite{uysal_turkeys_twitter_public_diplomacy_2019}. 
Twitter has been instrumental in studying human behavior with social media data 
\parencite{pfeffer_twitter_24_Hours_just_another_day_2023}, 
described as a digital social telescope by researchers in the social science field 
\parencite{mejova_twitter_2015_social_telescope}. 
It has provided a somewhat free environment and guided social and political debates to 
gain new dimensions \parencite{yerlikaya_social_media_manipulation_politics_2020}, 
where individual users can directly and publicly address comments to their representatives under 
conditions of anonymity \parencite{theocharis_twitter_political_incivility_2020}. The robust rise 
in Twitter's popularity has stemmed from increasing accessibility to technology and affordability. 
Millions of people consume news from social media sites like Twitter \parencite{anwar_analyzing_twitter_BERT_QAnon_2021}. 
In Turkey's case, Twitter began to be taken seriously after the unrest in the Middle East, 
especially after the Gezi Park protests in 2013 \parencite{zaharna_uysal_social_media_2016}, 
where Twitter was one of the most valuable media for protestor communication, given censorship 
\parencite{ogan__varol_twitter_gezi_park_2017}.
In 2020, Turkey was ranked as the 10th most-used language on Twitter, with around 560 thousand 
tweets posted daily \parencite{alshaabi_social_media_twitter_analysis_2021}. 

% Why should anyone care?
% What is the problem?

This thesis aims to analyze the Twitter data, provided by Sabanci University \parencite{secim2023},
to understand the Turkish Twitter discourse surrounding the May 2023 elections. Using innovative
topic modeling techniques, this thesis will discover the most prevalent topics in Turkish Twitter
between July 2022 and June 2023. It will uncover how these topics correlate with real-life events
and how they reflect the election agendas of parties. This thesis will also compare the results with 
results observed in other countries. In a non-English-speaking country like Turkey, this thesis 
furthermore seeks to find solutions to the need for a more thorough and data-driven 
analysis of political discussions on Turkish Twitter. 


In this chapter, the thesis starts by explaining the historical context and then continues 
to present the current political landscape. It demonstrates the importance of the May 2023 elections, 
emphasizes the significance of Twitter in Turkish politics, and deep dives into research questions. 
In the next chapter, the thesis examines various related works, asking similar questions and 
analyzing their results. After that, the thesis explains the Twitter dataset and used methodologies 
while collecting and analyzing the data. Next, the thesis deep dives into the analysis results, 
and later discusses the findings by interpreting them, highlighting both the limitations and future work. 
The final section summarizes the results and its implications.

\section{Background}

It is crucial to examine Turkey's historical political context to understand 
the complex political landscape and the May 2023 elections.

After the collapse of the Ottoman Empire, the Turkish Republic was declared in 1923.
Some attempts were made, but the first multi-party elections were held in 1946. 
Until 1945, the \ac{CHP} was the only party in the parliament, and until 1950 it was the ruling party.
The \ac{CHP} was founded by Mustafa Kemal Atatürk, also the founder of the Turkish Republic.

With a multi-party system in a young republic, political power was now open to various groups.
Different and new ideologies arose and started to organize politically \parencite{rise_of_islam_turkey}.
The military saw their role as the protector of the Republic and Atatürk's ideologies 
and overthrew the governments in 1960, 1971, and 1980.
The 1980 military coup, which introduced a new constitution, was after a period of
political fragmentation and civil instability in the 1970s. 

During the 1970s, political Islamism started to emerge, which challenged 
the secularist nationalism and modernization ideologies of the \ac{CHP} \parencite{erdoganism_akp_after_15_years}. 
Changes in the political structure, the constitution, and civil liberties, major economic crises in 1994 and 2001 \parencite{financial_crisis_turkey_1994}
contributed to Islamic political groups' political influence and strength, 
to the emergence of new political players and parties like the \ac{AKP} \parencite{rise_of_islam_turkey}.

Since 2002, \ac{AKP} has been in power in Turkey. Out of 15 elections, \ac{AKP} just lost the 
local elections in 2019, in which the opposition coalition won more than four significant municipalities. 
Especially in Istanbul, the opposition won twice because the first election was canceled. 
For the May 2023 elections, the main opposition coalition was established from \ac{CHP}, \ac{IyiP}, \ac{SAADET}, 
\ac{DP}, and two new parties were established out of \ac{AKP}: \ac{DEVA} and \ac{GP} 
\parencite{Atila_medyascope_tr_secim_tarihi_2022}. 
Even though most of the polls favored the opposition in the May 2023 elections 
\parencite{Saç_Çoban_teyit_anketler_2023}, \ac{AKP} has won the majority of the parliament and 
Recep Tayyip Erdogan was elected in the kickoff elections for the third time as president, after serving 
two terms as president and two terms as prime minister since 2003.
% mention: coalitions
% economic and political problems and polarization (Esen, 2023)
% democracy index
\section{Research Questions}

This section introduces the research questions guiding this thesis,
which are based on qualitative methods to analyze the Twitter discourse surrounding the May 2023 elections in Turkey.


The research questions are divided into two parts.
The first part will cover the main research objective of this thesis, 
which is the analysis of the topic modeling results. The first question is as follows:
``What were the most prevalent topics in Turkish Twitter discussions during the May 2023 elections?''.
This question is necessary to understand the main topics of the May 2023 elections discussed in social media.

The next question is ``How do real-life events during the election period correlate with shifts in discussion topics on Twitter, 
and in what ways do these shifts mirror political movements?''.
This question focuses on the reflection of real-life events and political movements in Twitter discussions. 

The third question is about parties and their election agendas: 
``How do the Twitter discussions about the ruling party and the opposition during the election lead-up reflect and compare 
to their respective election agendas and public statements?''.
This question is essential to understand the reflection of the election agendas of the parties and the differences between them on Twitter.

With these questions in mind, the second part of the research questions covers the comparison of the results of the topic modeling
with other research, where a similar approach was used for different countries.
The main question is as follows: 
``How do the key themes, content, and engagement levels in the Turkish Twitter discourse surrounding the May 2023 elections
 compare with those observed in the past elections in other countries?''.
