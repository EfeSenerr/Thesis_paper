% !TeX root = ../main.tex
% Add the above to each chapter to make compiling the PDF easier for some editors.

\chapter{Introduction}\label{chapter:introduction}
% why is this thesis important?
% why should anyone care?
% what is the problem?

It is crucial to understand Turkey's historical political context to understand
the complex political landscape and the May 2023 elections.
\section{Background}

After the collapse of the Ottoman Empire, the Turkish Republic was declared in 1923.
Some attempts were made, but the first multi-party elections were held in 1946. 
Until 1945, the \ac{CHP} was the only party in the parliament, and until 1950 it was the ruling party.
The \ac{CHP} was founded by Mustafa Kemal Atatürk, also the founder of the Turkish Republic.

With a multi-party system in a young republic, political power was now open to various groups.
Different and new ideologies arose and started to organize politically \parencite{rise_of_islam_turkey}.
The military saw their role as the protector of the republic and Atatürk's ideologies 
and overthrew the governments in 1960, 1971, and 1980.
The 1980 military coup, which introduced a new constitution, was after a period of
political fragmentation and civil instability in the 1970s. 

During the 1970s, political Islamism started to emerge, which challenged 
the secularist nationalism and modernization ideologies of the \ac{CHP} \parencite{erdoganism_akp_after_15_years}. 
Changes in the political structure, the constitution, and civil liberties, major economic crises in 1994 and 2001 \parencite{financial_crisis_turkey_1994}
contributed to Islamic political groups' political influence and strength, 
to the emergence of new political players and parties like the \ac{AKP} \parencite{rise_of_islam_turkey}.

% current political landscape
% May 2023 Elections
% Twitter


\section{Research Questions}

This chapter introduces the research questions guiding this thesis,
which are based on qualitative methods to analyze the Twitter discourse surrounding the May 2023 elections in Turkey.


The research questions are divided into two parts.
The first part will cover the main research objective of this thesis, 
which is the analysis of the topic modeling results. The first question is as follows:
``What were the most prevalent topics in Turkish Twitter discussions during the May 2023 elections?''.
This question is necessary to understand the main topics of the May 2023 elections discussed in social media.

The next question is ``How do real-life events during the election period correlate with shifts in discussion topics on Twitter, 
and in what ways do these shifts mirror political movements?''.
This question focuses on the reflection of real-life events and political movements in Twitter discussions. 

The third question is about parties and their election agendas: 
``How do the Twitter discussions about the ruling party and the opposition during the election lead-up reflect and compare 
to their respective election agendas and public statements?''.
This question is essential to understand the reflection of the election agendas of the parties and the differences between them on Twitter.

With these questions in mind, the second part of the research questions covers the comparison of the results of the topic modeling
with other research, where a similar approach was used for different countries.
The main question is as follows: 
``How do the key themes, content, and engagement levels in the Turkish Twitter discourse surrounding the May 2023 elections
 compare with those observed in the past elections in other countries?''.
