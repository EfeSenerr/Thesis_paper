% !TeX root = ../main.tex
% Add the above to each chapter to make compiling the PDF easier for some editors.

\chapter{Introduction}\label{chapter:introduction}

\section{Context}

Right now, I am testing if that works out.

\section{Research Questions}

This chapter introduces the research questions guiding this thesis,
which are based on qualitative methods to analyze the Twitter discourse surrounding the May 2023 elections in Turkey.


The research questions are divided into two parts.
The first part will cover the main research objective of this thesis, 
which is the analysis of the topic modeling results. The first question is as follows:
``What were the most prevalent topics in Turkish Twitter discussions during the May 2023 elections?''.
This question is necessary to understand the main topics of the May 2023 elections discussed in social media.

The next question is ``How do real-life events during the election period correlate with shifts in discussion topics on Twitter, 
and in what ways do these shifts mirror political movements?''.
This question focuses on the reflection of real-life events and political movements in Twitter discussions. 

The third question is about parties and their election agendas: 
``How do the Twitter discussions about the ruling party and the opposition during the election lead-up reflect and compare 
to their respective election agendas and public statements?''.
This question is essential to understand the reflection of the election agendas of the parties and the differences between them on Twitter.

With these questions in mind, the second part of the research questions covers the comparison of the results of the topic modeling
with other research, where a similar approach was used for different countries.
The main question is as follows: 
``How do the key themes, content, and engagement levels in the Turkish Twitter discourse surrounding the May 2023 elections
 compare with those observed in the past elections in other countries?''.
