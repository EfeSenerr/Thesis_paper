\chapter{\abstractname}

This thesis dives into the depths of Turkish Twitter political discourse around the May 2023 
general elections, collecting and analyzing around 150 million tweets from July 2022 to June 2023. 
BERTopic is leveraged as the neural topic model to uncover the trending themes. A base model is 
trained on a 1\% random sample, with the remaining data transformed according to the base model's 
topic allocations.

To discover the key themes of the political discourse on Twitter, over 1000 topic clusters are 
revealed initially. The trending topics, their representative words, and counts are presented 
through tables, graphs, and other visualizations. The thesis discovers that topics associated with 
the government engage with nationalism, criticism and praise of political figures, demands and 
wishes from the government, and President Erdoğan's reelection chance. On the other hand, 
opposition-related topics focus on presidential candidacy, coalition dynamics, biased judiciary, 
the economic situation, and disinformation against the opposition block.

This thesis provides insights into the correlation between real-life events and trending topics on 
Twitter, where the political discourse on Twitter and the coalition strategies, public speeches, 
and announcements are also compared and discussed. Furthermore, the thesis performs a comparative 
analysis by contrasting the results with other research analyzing different elections and political 
domains.

In conclusion, this thesis contributes to the academic environment by pioneering one of the 
first topic modeling analyses on Turkish political discourse utilizing big data, which sheds 
light on the May 2023 general elections and aims to better understand Turkish political discourse 
and motivate future research by presenting the methodologies and findings.